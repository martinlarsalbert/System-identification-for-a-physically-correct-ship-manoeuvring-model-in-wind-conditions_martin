% Move 1 - Background/introduction/situation

System identification offers ways to identify proper models to describe a ship's dynamics in real operational conditions, but also poses some major challenges, such as multicollinearity and generality of the identified model. 
% Move 2 - Present research/purpose
This paper proposes a new physics informed manoeuvring model, where a deterministic semi-empirical rudder model has been added, to guide the identification towards a physically correct model.  
This is an essential building block to solve ship manoeuvring modelling uncertainties from wind, waves, and currents, which are either added for a real sea conditions or important to model for ships with wind-assisted propulsion.
In the physics informed manoeuvring modeling framework, a systematical procedure is developed to establish various force/motion components within the manoeuvring system by the inverse dynamics regression. 
% Move 3 - Methods/materials/subjects/procedures
The novel test case wind powered pure car carrier (WPCC) is used to assess the physical correctness. First, a reference model, assumed to resemble the physically correct kinetics, is established via parameter identification on virtual captive zigzag tests. Then, the model tests are used to build both the physics informed model and the uninformed mathematical Abkowitz model for comparison.


% Move 4 - Results/findings
All of the models predicted the zigzag tests with satisfactory agreement and can thus indeed be considered as being mathematically correct; However, the introduction of a semi-empirical rudder model seems to have guided the identification towards a more physically correct calm water hydrodynamic model, with lower multicollinearity and better generalization. The physics informed model predicted forces and moments that were much more in agreement to the reference model than the Abkowitz model did, and can thus be considered as the more physically correct model. 

% Move 5 - Discussion/conclusion/significance